\documentclass[11pt]{article}

\usepackage{amsmath}
\usepackage{textcomp}
\usepackage[margin=2.5cm]{geometry}
% Add other packages here %


% Put your group number and names in the author field %
\title{\bf Exercise 4\\ Implementing a centralized agent}
\author{Group 39 : Hugo Bonnome, Pedro Amorim}


% N.B.: The report should not be longer than 3 pages %


\begin{document}
\maketitle

\section{Solution Representation}
Unless noted otherwise, values that are not defined explicitely here use the
definitions provided in the reference paper. 

\subsection{Variables}
% Describe the variables used in your solution representation %
The following variables are used to define the CSP:
\begin{itemize}
\item $P$ - a set of plans, one for each vehicle.
  $$P = \{p_1, p_2,\cdots, p_{N_V}\}$$
\item $p_i$ - a doubly linked list representing the sequence of actions in a
  particular plan i.e. the deliveries and pickups that the vehicle should
  execute.
  $$p_i = (u_1 \rightarrow u_2 \rightarrow d_2 \rightarrow \cdots | u_i \in U
  \wedge d_i \in D)$$
\item $U$ - the set of pickups associated to each task.
  $$U = \{u_1, u_2, \cdots, u_{N_T}\}$$
\item $D$ - the set of deliveries associated to each task.
  $$D = \{d_1, d_2, \cdots, d_{N_T}\}$$
\end{itemize}

\subsection{Constraints}
% Describe the constraints in your solution representation %
\begin{enumerate}
\item All tasks must be delivered i.e. $p_1 \cup p_2 \cup \cdots \cup p_{N_V} = U
  \cup D$ 
\item Max load of a given plan must be under the carrying capacity of the
  vehicle i.e. $maxLoad(p_i) \leq maxCapacity(v_i)\  \forall_{p_i \in P}$
\item A plan can only deliver a task it has previously picked up i.e. $u_i \in
  predecessors(d_i)\ \forall_{u_i, d_i \in p_i}$
\item A given action can only appear once in the set of all the plans.
\item If there is a pickup action in a plan then there is a corresponding
delivery action in the same plan and vice versa. i.e. $u_i \in p_i
\leftrightarrow d_i \in p_i \ \forall_{p_i \in P,\  u_i \in p_i,\  d_i \in p_i}$
\end{enumerate}

\subsection{Objective function}
% Describe the function that you optimize %
Let us first define the total cost of a vehicle $v$ with a given plan. $$c_v =
\sum_{a_i \in p_v} (dist(a_i))\cdot cost(v)$$
The total cost of the company is then defined by the sum of the costs of each
vehicle in the company i.e.
$$\sum_{v_i \in V} c_{v_i}$$

\section{Stochastic optimization}

\subsection{Initial solution}
% Describe how you generate the initial solution %

\subsection{Generating neighbours}
% Describe how you generate neighbors %

\subsection{Stochastic optimization algorithm}
% Describe your stochastic optimization algorithm %


\section{Results}

\subsection{Experiment 1: Model parameters}
% if your model has parameters, perform an experiment and analyze the results for different parameter values %

\subsubsection{Setting}
% Describe the settings of your experiment: topology, task configuration, number of tasks, number of vehicles, etc. %
% and the parameters you are analyzing %

\subsubsection{Observations}
% Describe the experimental results and the conclusions you inferred from these results %

\subsection{Experiment 2: Different configurations}
% Run simulations for different configurations of the environment (i.e. different tasks and number of vehicles) %

\subsubsection{Setting}
% Describe the settings of your experiment: topology, task configuration, number of tasks, number of vehicles, etc. %

\subsubsection{Observations}
% Describe the experimental results and the conclusions you inferred from these results %
% Reflect on the fairness of the optimal plans. Observe that optimality requires some vehicles to do more work than others. %
% How does the complexity of your algorithm depend on the number of vehicles and various sizes of the task set? %

\end{document}