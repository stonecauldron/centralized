\documentclass[11pt]{article}

\usepackage{amsmath}
\usepackage{textcomp}
\usepackage[margin=2.5cm]{geometry}
% Add other packages here %

% Put your group number and names in the author field %
\title{\bf Exercise 4\\ Implementing a centralized agent}
\author{Group 39 : Hugo Bonnome, Pedro Amorim}


% N.B.: The report should not be longer than 3 pages %


\begin{document}
\maketitle

\section{Solution Representation}
Unless noted otherwise, values that are not defined explicitely here use the
definitions provided in the reference paper. 

\subsection{Variables}
% Describe the variables used in your solution representation %
The following variables are used to define the CSP:
\begin{itemize}
\item $P$ - a set of plans, one for each vehicle.
  $$P = \{p_1, p_2,\cdots, p_{N_V}\}$$
\item $p_i$ - a doubly linked list representing the sequence of actions in a
  particular plan i.e. the deliveries and pickups that the vehicle should
  execute.
  $$p_i = (u_1 \rightarrow u_2 \rightarrow d_2 \rightarrow \cdots | u_i \in U
  \wedge d_i \in D)$$
\item $U$ - the set of pickups associated to each task.
  $$U = \{u_1, u_2, \cdots, u_{N_T}\}$$
\item $D$ - the set of deliveries associated to each task.
  $$D = \{d_1, d_2, \cdots, d_{N_T}\}$$
\end{itemize}

\subsection{Constraints}
% Describe the constraints in your solution representation %
\begin{enumerate}
\item All tasks must be delivered i.e. $p_1 \cup p_2 \cup \cdots \cup p_{N_V} = U
  \cup D$ 
\item Max load of a given plan must be under the carrying capacity of the
  vehicle i.e. $maxLoad(p_i) \leq maxCapacity(v_i)\  \forall_{p_i \in P}$
\item A plan can only deliver a task it has previously picked up i.e. $u_i \in
  predecessors(d_i)\ \forall_{u_i, d_i \in p_i}$
\item A given action can only appear once in the set of all the plans.
\item If there is a pickup action in a plan then there is a corresponding
delivery action in the same plan and vice-versa. i.e. $u_i \in p_i
\leftrightarrow d_i \in p_i \ \forall_{p_i \in P,\  u_i \in p_i,\  d_i \in p_i}$
\end{enumerate}

\subsection{Objective function}
% Describe the function that you optimize %
Let us first define the total cost of a vehicle $v$ with a given plan. $$c_v =
\sum_{a_i \in p_v} (dist(a_i))\cdot cost(v)$$
The total cost of the company is then defined by the sum of the costs of each
vehicle in the company i.e.
$$\sum_{v_i \in V} c_{v_i}$$

\section{Stochastic optimization}

\subsection{Initial solution}
% Describe how you generate the initial solution %
The initial solution is generated simply by giving all the tasks to the biggest
vehicle.

\subsection{Generating neighbours}
% Describe how you generate neighbors %
Two operators are used to generate neighbours:
\begin{itemize}
\item \textit{Changing vehicle}: give a pair of delivery and pickup tasks to
  another vehicle by appending them to the plan of the other vehicle. Weight
  constraints should not be violated by this change. 
\item \textit{Changing task order}: change the order of two tasks in the plan of
  a vehicle while making sure that the order does not violate constraint $3$;
  namely the fact that a delivery task cannot come before the associated pickup
  task.
\end{itemize}

\subsection{Stochastic optimization algorithm}
% Describe your stochastic optimization algorithm %
The algorithms works by first generating an initial solution that satisfies all
the constraints. Then locally similar solutions are generated and the best one
among them is chosen. This is repeated until a given number of iterations is
reached. Since it uses an hill climbing heuristic, the algorithm is susceptible
to getting stuck in a local minima and thus there is no guarantee of finding an
optimal solution. To reduce the chances of getting stuck in a local minima,
local improvements are only chosen with a certain probability.

\section{Results}

\subsection{Experiment 1: Parameter $p$}
% if your model has parameters, perform an experiment and analyze the results for different parameter values %

\subsubsection{Setting}
% Describe the settings of your experiment: topology, task configuration, number of tasks, number of vehicles, etc. %
% and the parameters you are analyzing %
\begin{itemize}
\item Topology: England.
\item Task distribution: uniform probability, constant reward and constant
  weight.
\item Task number: 30.
\item Vehicles : 4 vehicles with all the same characteristics.
\item $p \in \{0, 0.8, 1\}$
\item Number of iterations: 20000.
\end{itemize}

\subsubsection{Observations}
% Describe the experimental results and the conclusions you inferred from these results %
\begin{itemize}
\item With $p = 0$ the total cost of the plan is equal to: 58353.0 CHF. This can
be explained by the fact that the solution is chosen in a completely random
manner as long as it satisfies the constraints. It is thus far from being
efficient.

\item With $p = 0.8$ the total cost of the plan is equal to: 18518.0 CHF. This
  is a much better result and it can easily explained by the fact that the
  algorithm can now do an hill-ascent on the range of solutions.

\item With $p = 1$ the total cost of the plan is equal to: 28943.6 CHF. The
  higher cost can be explained by the fact that the algorithm will now almost
  always converge to a local minimum since the random changes in the solution no
  longer occur.
\end{itemize}

\subsection{Experiment 2: Different vehicle numbers}
% Run simulations for different configurations of the environment (i.e. different tasks and number of vehicles) %

\subsubsection{Setting}
% Describe the settings of your experiment: topology, task configuration, number of tasks, number of vehicles, etc. %
Same setting as experiment 1; the difference lies in the number of vehicles $N_V
\in \{1, 4, 10\}$ and $p = 0.8$

\subsubsection{Observations}
% Describe the experimental results and the conclusions you inferred from these results %
% Reflect on the fairness of the optimal plans. Observe that optimality requires some vehicles to do more work than others. %
% How does the complexity of your algorithm depend on the number of vehicles and various sizes of the task set? %
\begin{itemize}
\item With $N_V = 1$ the total cost is equal to 28945.6 CHF. We get the same
  result as in experiment 1 when $p = 1$. This can be understood by the fact
  that the algorithm computes the best order of delivery when only one vehicle
  is available.
\item With $N_V = 4$ the total cost is equal to 18518.5 CHF. In this setting the
  vehicles share the task set and as such can be more efficient than a lone
  vehicle. This does not mean however that the tasks are distributed equally
  since the only metric of performance is the total cost.
\item With $N_V = 10$ the total cost is equal to 30237.5 CHF. We can see that
  having more vehicles does not mean having a lower cost. In this setting many
  vehicles do not even act for the whole duration of the simulation. This is
  understandable by the fact that what is being optimized here is the total cost
  and not the speed of delivery of all tasks.
\end{itemize}
The time complexity of the algorithm grows linearly with the number of vehicles
present in the simulation and the number of tasks. This comes from the fact that
the \textit{generateNeighbours} function has two operations:
\begin{enumerate}
\item Changing the order of tasks within the plan of a given vehicle. This
  operation grows linearly with the number of tasks. 
\item Giving a specific task to another vehicle. This grows linearly with the
  number of vehicles to consider.
\end{enumerate}
\end{document}